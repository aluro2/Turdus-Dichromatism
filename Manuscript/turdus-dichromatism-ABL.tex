% Options for packages loaded elsewhere
\PassOptionsToPackage{unicode}{hyperref}
\PassOptionsToPackage{hyphens}{url}
\PassOptionsToPackage{dvipsnames,svgnames*,x11names*}{xcolor}
%
\documentclass[
  a4paper,
]{article}
\usepackage{amsmath,amssymb}
\usepackage{lmodern}
\usepackage{setspace}
\usepackage{ifxetex,ifluatex}
\ifnum 0\ifxetex 1\fi\ifluatex 1\fi=0 % if pdftex
  \usepackage[T1]{fontenc}
  \usepackage[utf8]{inputenc}
  \usepackage{textcomp} % provide euro and other symbols
\else % if luatex or xetex
  \usepackage{unicode-math}
  \defaultfontfeatures{Scale=MatchLowercase}
  \defaultfontfeatures[\rmfamily]{Ligatures=TeX,Scale=1}
  \setmainfont[]{Lato}
\fi
% Use upquote if available, for straight quotes in verbatim environments
\IfFileExists{upquote.sty}{\usepackage{upquote}}{}
\IfFileExists{microtype.sty}{% use microtype if available
  \usepackage[]{microtype}
  \UseMicrotypeSet[protrusion]{basicmath} % disable protrusion for tt fonts
}{}
\usepackage{xcolor}
\IfFileExists{xurl.sty}{\usepackage{xurl}}{} % add URL line breaks if available
\IfFileExists{bookmark.sty}{\usepackage{bookmark}}{\usepackage{hyperref}}
\hypersetup{
  pdftitle={Rapid species recognition favors greater avian-perceived plumage dichromatism in true thrushes (genus: Turdus)},
  pdfauthor={Alec B. Luro1*, Mark E. Hauber1},
  colorlinks=true,
  linkcolor=blue,
  filecolor=Maroon,
  citecolor=blue,
  urlcolor=Blue,
  pdfcreator={LaTeX via pandoc}}
\urlstyle{same} % disable monospaced font for URLs
\usepackage[margin=1in]{geometry}
\setlength{\emergencystretch}{3em} % prevent overfull lines
\providecommand{\tightlist}{%
  \setlength{\itemsep}{0pt}\setlength{\parskip}{0pt}}
\setcounter{secnumdepth}{-\maxdimen} % remove section numbering
\usepackage[left]{lineno}
\linenumbers
\usepackage{graphicx}
\usepackage{booktabs}
\usepackage{caption}
\usepackage{lscape}
\ifluatex
  \usepackage{selnolig}  % disable illegal ligatures
\fi
\newlength{\cslhangindent}
\setlength{\cslhangindent}{1.5em}
\newlength{\csllabelwidth}
\setlength{\csllabelwidth}{3em}
\newenvironment{CSLReferences}[2] % #1 hanging-ident, #2 entry spacing
 {% don't indent paragraphs
  \setlength{\parindent}{0pt}
  % turn on hanging indent if param 1 is 1
  \ifodd #1 \everypar{\setlength{\hangindent}{\cslhangindent}}\ignorespaces\fi
  % set entry spacing
  \ifnum #2 > 0
  \setlength{\parskip}{#2\baselineskip}
  \fi
 }%
 {}
\usepackage{calc}
\newcommand{\CSLBlock}[1]{#1\hfill\break}
\newcommand{\CSLLeftMargin}[1]{\parbox[t]{\csllabelwidth}{#1}}
\newcommand{\CSLRightInline}[1]{\parbox[t]{\linewidth - \csllabelwidth}{#1}\break}
\newcommand{\CSLIndent}[1]{\hspace{\cslhangindent}#1}

\title{Rapid species recognition favors greater avian-perceived plumage
dichromatism in true thrushes (genus: \emph{Turdus})}
\author{Alec B. Luro\textsuperscript{1}*, Mark E.
Hauber\textsuperscript{1}}
\date{\textsuperscript{1} Department of Evolution, Ecology and Behavior,
School of Integrative Biology, University of Illinois at
Urbana-Champaign *alec.b.luro@mail.com}

\begin{document}
\maketitle

\setstretch{1.5}
\hypertarget{abstract}{%
\section{Abstract}\label{abstract}}

\hypertarget{keywords}{%
\subsection{Keywords}\label{keywords}}

\emph{dichromatism}, \emph{plumage}, \emph{species recognition}

\hypertarget{background}{%
\section{Background}\label{background}}

\hypertarget{methods}{%
\section{Methods}\label{methods}}

\hypertarget{plumage-sexual-dichromatism}{%
\subsection{\texorpdfstring{\emph{Plumage sexual
dichromatism}}{Plumage sexual dichromatism}}\label{plumage-sexual-dichromatism}}

A total of N=77 \emph{Turdus} thrush species (approximately
\textasciitilde89\% of all known true thrush species) were sampled for
plumage spectral reflectance using prepared bird skin specimens at the
American Museum of Natural History in New York City and the Field Museum
in Chicago. Reflectance measurements spanning 300-700nm were taken in
triplicate from the belly, breast, throat, crown and mantle plumage
patches {[}\protect\hyperlink{ref-andersson2006}{1}{]} of each
individual. N=3 male and N=3 female individuals were measured for most
species (exceptions: \emph{T. lawrencii}, N=2 males and N=2 females;
\emph{T. swalesi}, N=1 male and N=1 female). Reflectance spectra were
measured using a 400 μm fiber optic reflection probe fitted with a
rubber stopper to maintain a consistent measuring distance of 3 mm and
area of 2 mm2 at a 90° angle to the surface of the feather patch.
Measurements were taken using a JAZ spectrometer with a pulsed-xenon
light source (Ocean Optics, Dunedin, USA) and we used a diffuse 99\%
reflectance white standard (Spectralon WS-1-SL, Labsphere, North Sutton
NH, USA).

We applied a receptor-noise limited visual model
{[}\protect\hyperlink{ref-vorobyev1998}{2}{]} of the European Blackbird
(\emph{T. merula}) visual system
{[}\protect\hyperlink{ref-hart2000}{3}{]} in the \emph{pavo}
{[}\protect\hyperlink{ref-maia2019}{4}{]}⁠ package in R v4.0.0
{[}\protect\hyperlink{ref-rcoreteam2020}{5}{]}⁠ to calculate
avian-perceived chromatic and achromatic visual contrast (in units of
``Just-Noticeable Differences'',or JNDs) of male vs.~female plumage
patches for all sampled \emph{Turdus} species. Chromatic and achromatic
JNDs were calculated for male-female pairs within each species (i.e.,
N=9 JND values calculated per patch for each species where N=3 males and
N=3 females sampled), and then JND values were averaged for each
species' respective plumage patches. Under ideal laboratory conditions,
1 JND is generally considered to be the discriminable threshold past
which an observer is predicted to be able to perceive the two colors as
different. However, natural light environments vary both spatially and
temporally {[}\protect\hyperlink{ref-endler1993}{6}{]}⁠, bringing into
question the accuracy of a 1 JND threshold for generalizing visual
contrast under natural conditions. Therefore, we calculated the total
number of sexually-dichromatic plumage patches per species (out of N=5
measured patches) as the number of plumage patches with average JND
values \textgreater{} 1, 2, or 3 to account for uncertainty in visual
discrimination thresholds due to variation in psychophysical and ambient
lighting conditions affecting the strength of between-sex plumage visual
contrast {[}\protect\hyperlink{ref-kemp2015}{7}{]}⁠.

\hypertarget{life-history-data}{%
\subsection{\texorpdfstring{\emph{Life History
Data}}{Life History Data}}\label{life-history-data}}

\hypertarget{breeding-timing-model}{%
\subsubsection{\texorpdfstring{\emph{Breeding Timing
Model}}{Breeding Timing Model}}\label{breeding-timing-model}}

We collected data on migration behavior and breeding season length from
\emph{Thrushes} {[}\protect\hyperlink{ref-clement2000}{8}{]} and the
\emph{Handbook of the Birds of the World}
{[}\protect\hyperlink{ref-delhoyo2017}{9}{]}⁠. We assigned three
different kinds of migratory behavior: 1) \emph{full migration} when a
species description clearly stated that a species ``migrates'', 2)
\emph{partial migration} when a species was described to have
``altitudinal migration'', ``latitudinal migration'' or ``movement
during non-breeding season'', or 3) \emph{sedentary} when when a species
was described as ``resident'' or ``sedentary''. Breeding season length
was defined as the number of months the species breeds each year.

\hypertarget{breeding-sympatry-model}{%
\subsubsection{\texorpdfstring{\emph{Breeding Sympatry
Model}}{Breeding Sympatry Model}}\label{breeding-sympatry-model}}

Species' breeding ranges were acquired from \emph{BirdLife
International}
{[}\protect\hyperlink{ref-birdlifeinternationalandhandbookofthebirdsoftheworld2018}{10}{]}⁠.
We calculated congener breeding range overlaps (as percentages) using
the \emph{letsR} package in R
{[}\protect\hyperlink{ref-vilela2015}{11}{]}⁠. We then calculated the
number of sympatric species as the number of congeners with breeding
ranges that overlap \textgreater30\% with the focal species' breeding
range {[}\protect\hyperlink{ref-cooney2017}{12}{]}.

\hypertarget{breeding-spacing-model}{%
\subsubsection{\texorpdfstring{\emph{Breeding Spacing
Model}}{Breeding Spacing Model}}\label{breeding-spacing-model}}

Species' breeding range sizes (in km2) were acquired using the
\emph{BirdLife International} breeding range maps. Species' island
vs.~mainland residence was also determined using breeding ranges from
\emph{BirdLife International}. Mainland residence was assigned if the
species had a breeding range on any continent and Japan. Island
residence was assigned to species having a breeding range limited to a
non-continental landmass entirely surrounded by an oceanic body of
water.

\hypertarget{statistical-modeling}{%
\subsection{\texorpdfstring{\emph{Statistical
Modeling}}{Statistical Modeling}}\label{statistical-modeling}}

We used phylogenetically-corrected Bayesian multilevel logistic
regression models using the \emph{brms} v2.13.0 package
{[}\protect\hyperlink{ref-burkner2017}{13}{]} in R v4.0.0
{[}\protect\hyperlink{ref-rcoreteam2020}{5}{]}⁠ where responses, the
number of sexually-dichromatic patches \textgreater1, 2, and 3 chromatic
and achromatic JNDs, were modeled as binomial trials (N=5 plumage patch
``trials'') to test for associations with breeding timing, breeding
sympatry and breeding spacing. For all phylogenetically-corrected
models, we used the \emph{Turdus} phylogeny from Nylander et al.~(2008)
{[}\protect\hyperlink{ref-nylander2008}{14}{]}to create a covariance
matrix of species' phylogenetic relationships. All models used a dataset
of N=67 out of the \emph{Turdus} species for which all the types of data
(see above) were available.

Our \emph{breeding timing} models included the following predictors:
z-scores of breeding season length (mean centered and divided by one
standard deviation), migratory behavior (full migration as the reference
category versus partial migration or sedentary), and their interaction.
\emph{Breeding sympatry} models included the number of sympatric species
with greater than 30\% breeding range overlap as the only predictor of
the probability of having a sexually-dichromatic plumage patch.
\emph{Breeding spacing} models included \(log_{e}\) transformed breeding
range size (km2) and breeding landmass (mainland as the reference
category versus island). We also ran null models (intercept only) for
all responses. All models' intercepts and response standard deviations
were assigned a weak prior (Student T: df = 3, location = 0, scale =
10), and predictor coefficients were assigned flat priors. We ran each
model for 6,000 iterations across 6 chains and assessed Markov Chain
Monte Carlo (MCMC) convergence using the Gelman-Rubin diagnostic (Rhat)
{[}\protect\hyperlink{ref-gelman2013}{15}{]}. We then performed k-fold
cross-validation {[}\protect\hyperlink{ref-vehtari2017}{16}{]} to refit
each model \emph{K}=16 times. For each k-fold, the training dataset
included a randomly selected set of \(N- N\frac{1}{K}\) or N≈63 species,
and the testing dataset included \(N\frac{1}{K}\) or N≈4 species not
included in the training dataset. Finally, we compared differences
between the models' expected log pointwise predictive densities (ELPD)
to assess which model(s) best predicted the probability of having a
sexually-dichromatic plumage patch.
{[}\protect\hyperlink{ref-vehtari2017}{16}{]}⁠.

Models' predictor effects were assessed using 90\% highest-density
intervals of the posterior distributions and probability of direction,
the proportion of the posterior distribution that shares the same sign
(positive or negative) as the posterior median
{[}\protect\hyperlink{ref-makowski2019}{17}{]}, to provide estimates of
the probability of that a predictor has an entirely positive or negative
effect on the presence of sexually-dimorphic plumage patches.

\hypertarget{results}{%
\section{Results}\label{results}}

We obtained N ≥ 4000 effective samples for each model parameter and all
models' Markov Chains (MCMC) successfully converged (Rhat = 1 for all
models' parameters) (Supplementary Figure). All \emph{breeding
sympatry}, \emph{breeding timing}, and \emph{breeding spacing} models
performed similarly well and substantially better than \emph{intercept
only} models in predicting the probability of having a sexually
dimorphic plumage patch with achromatic JND values \textgreater{} 1, 2,
or 3 (Table 1; all models predicting achromatic plumage patches had ELPD
values within 4, following the convention of Burnham and Anderson
(2002){[}\protect\hyperlink{ref-burnham2002}{18}{]}). Among models
predicting the probability of having a sexually-dichromatic plumage
patch with chromatic JND values \textgreater1, 2, or 3, all models
performed much better than \emph{intercept only} models, and
\emph{breeding sympatry} models had the best predictive performance
(Table 1; \emph{breeding sympatry} models all have ELPD =0, only the
\emph{breeding spacing} models predicting dichromatic plumage patches
with had similar predictive performance).

Among predictors of achromatically sexually-dimorphic plumage patches

\begin{table}[!h]

\caption{\label{tab:table01}Expected log pointwise predictive densities (ELPD) differences and
kfold information criterion values of models (ELPD Difference ± standard error (kfold IC ± standard error)). Lower values indicate greater model prediction performance.}
\centering
\resizebox{\linewidth}{!}{
\renewcommand{\arraystretch}{1.5}
\begin{tabular}[t]{llllll}
\toprule
\multicolumn{1}{l}{} & \multicolumn{1}{l}{} & \multicolumn{4}{l}{Model} \\
\cmidrule(l{3pt}r{3pt}){3-6}
Plumage Metric & JND Threshold & Breeding Sympatry & Breeding Timing & Breeding Spacing & Intercept Only\\
\midrule
\addlinespace[0.3em]
\multicolumn{6}{l}{\textbf{Achromatic}}\\
\hspace{1em} & 1 JND & 0 ± 0 (-122.17 ± 0.67) & -2.51 ± 2.49 (-124.68 ± 2.38) & -2.59 ± 1.01 (-124.76 ± 1.04) & -21.69 ± 7.36 (-143.87 ± 7.51)\\
\hspace{1em} & 2 JND & 0 ± 0 (-98.94 ± 7.56) & -1.19 ± 3.95 (-100.13 ± 9.22) & -0.7 ± 1.34 (-99.64 ± 7.92) & -52.42 ± 12.67 (-151.36 ± 13.4)\\
\hspace{1em} & 3 JND & -0.04 ± 1.4 (-85.4 ± 8.91) & -1.7 ± 4.41 (-87.07 ± 10.71) & 0 ± 0 (-85.37 ± 8.76) & -28.54 ± 10.02 (-113.91 ± 13.65)\\
\addlinespace[0.3em]
\multicolumn{6}{l}{\textbf{Chromatic}}\\
\hspace{1em} & 1 JND & 0 ± 0 (-115.75 ± 2.95) & -5.67 ± 3.55 (-121.42 ± 2.28) & -2.73 ± 3.4 (-118.49 ± 2.67) & -14.8 ± 7.22 (-130.55 ± 7.05)\\
\hspace{1em} & 2 JND & 0 ± 0 (-88.47 ± 8.77) & -3.8 ± 4.46 (-92.27 ± 10.01) & -3.32 ± 5.29 (-91.79 ± 10.91) & -50.53 ± 14.49 (-139 ± 16.77)\\
\hspace{1em} & 3 JND & 0 ± 0 (-62.77 ± 10.41) & -8 ± 4.32 (-70.77 ± 12.29) & -4.43 ± 3.9 (-67.2 ± 11.72) & -47.63 ± 15.34 (-110.4 ± 20.01)\\
\bottomrule
\end{tabular}}
\end{table}

\newpage
\begin{landscape}
\begin{table}

\caption{\label{tab:table02}Model predictor effect estimates (posterior median log-odds and 90\% highest-density interval) on the
  presence of a plumage patch with achromatic or chromatic visual contrast values $>$
  1, 2, and 3 JND. Model effects with a probability of direction (pd) value $\geq$ 0.90
  are bolded in \textcolor{red}{\textbf{red}} for a negative effect and \textcolor{blue}{\textbf{blue}} for a positive effect on
  plumage dichromatism.}
\centering
\resizebox{\linewidth}{!}{
\renewcommand{\arraystretch}{1.5}
\begin{tabular}[t]{llllllll}
\toprule
Model & Parameter & Achromatic, 1 JND & Achromatic, 2 JND & Achromatic, 3 JND & Chromatic, 1 JND & Chromatic, 2 JND & Chromatic, 3 JND\\
\midrule
\addlinespace[0.3em]
\multicolumn{1}{l}{\textbf{Breeding Timing}}\\
 & Intercept & \textcolor{black}{-1.16 (-3.87, 1.67), pd = 0.76} & \textcolor{red}{\textbf{-8.36 (-16.28, -0.62), pd = 0.98}} & \textcolor{red}{\textbf{-7.81 (-14.83, -1.66), pd = 0.99}} & \textcolor{black}{-0.88 (-2.98, 1.03), pd = 0.78} & \textcolor{red}{\textbf{-7.21 (-15.29, 0.55), pd = 0.95}} & \textcolor{red}{\textbf{-12.71 (-28.03, 0.31), pd = 0.96}}\\
 & Breeding Season Length & \textcolor{black}{-0.06 (-0.62, 0.56), pd = 0.57} & \textcolor{red}{\textbf{-2.26 (-4.72, 0.05), pd = 0.97}} & \textcolor{red}{\textbf{-1.39 (-3.56, 0.4), pd = 0.91}} & \textcolor{black}{-0.12 (-0.59, 0.34), pd = 0.66} & \textcolor{red}{\textbf{-1.99 (-4.64, 0.35), pd = 0.94}} & \textcolor{black}{-2.5 (-8.12, 2.21), pd = 0.83}\\
 & Partial Migration vs. No Migration & \textcolor{black}{-0.04 (-1.16, 1.01), pd = 0.53} & \textcolor{black}{1.41 (-1.2, 4.12), pd = 0.83} & \textcolor{black}{1.29 (-0.82, 3.57), pd = 0.85} & \textcolor{blue}{\textbf{0.79 (-0.06, 1.59), pd = 0.94}} & \textcolor{black}{1.9 (-0.87, 4.9), pd = 0.88} & \textcolor{blue}{\textbf{4.26 (-1.13, 10.99), pd = 0.92}}\\
 & Full Migration vs. No Migration & \textcolor{blue}{\textbf{1.6 (-0.05, 3.19), pd = 0.96}} & \textcolor{blue}{\textbf{4.2 (1.16, 7.5), pd = 0.99}} & \textcolor{blue}{\textbf{3.11 (0.46, 5.73), pd = 0.98}} & \textcolor{black}{0.83 (-0.37, 1.99), pd = 0.88} & \textcolor{blue}{\textbf{4.39 (1.03, 8.14), pd = 0.99}} & \textcolor{blue}{\textbf{5.46 (-0.68, 12.61), pd = 0.95}}\\
 & Breeding Season Length x Partial Migration & \textcolor{black}{0.29 (-0.73, 1.37), pd = 0.68} & \textcolor{blue}{\textbf{3.03 (-0.14, 6.38), pd = 0.96}} & \textcolor{blue}{\textbf{2.11 (-0.27, 4.69), pd = 0.94}} & \textcolor{black}{0.33 (-0.43, 1.14), pd = 0.76} & \textcolor{blue}{\textbf{2.2 (-0.82, 5.53), pd = 0.9}} & \textcolor{black}{3.54 (-2.58, 11.13), pd = 0.85}\\
 & Breeding Season Length x Full Migration & \textcolor{blue}{\textbf{1.58 (-0.4, 3.68), pd = 0.9}} & \textcolor{blue}{\textbf{4.19 (-0.53, 9.34), pd = 0.93}} & \textcolor{black}{2.8 (-1.3, 6.72), pd = 0.89} & \textcolor{black}{0.52 (-1.16, 2.12), pd = 0.7} & \textcolor{blue}{\textbf{5.08 (-0.18, 11.12), pd = 0.95}} & \textcolor{black}{6.07 (-4.27, 17.43), pd = 0.85}\\
 & Phylogenetic Signal λ, Median (90\% Credible Interval) & \textcolor{black}{0.29 (0.16, 0.43)} & \textcolor{black}{0.72 (0.56, 0.86)} & \textcolor{black}{0.61 (0.42, 0.8)} & \textcolor{black}{0.17 (0.08, 0.28)} & \textcolor{black}{0.74 (0.57, 0.88)} & \textcolor{black}{0.89 (0.77, 0.97)}\\
\addlinespace[0.3em]
\multicolumn{1}{l}{\textbf{Breeding Spacing}}\\
\hspace{1em} & Intercept & \textcolor{black}{-1.94 (-6.01, 2.01), pd = 0.8} & \textcolor{red}{\textbf{-9.77 (-20.11, 0.89), pd = 0.95}} & \textcolor{red}{\textbf{-10.31 (-19.2, -1.98), pd = 0.98}} & \textcolor{black}{-0.67 (-3.63, 2.27), pd = 0.65} & \textcolor{red}{\textbf{-8.32 (-18.86, 2.03), pd = 0.92}} & \textcolor{red}{\textbf{-12.87 (-30.57, 4.41), pd = 0.91}}\\
\hspace{1em} & Island vs. Mainland & \textcolor{black}{0.08 (-1.38, 1.57), pd = 0.54} & \textcolor{black}{-0.64 (-4.43, 2.88), pd = 0.61} & \textcolor{black}{-0.09 (-3.02, 2.96), pd = 0.52} & \textcolor{red}{\textbf{-1.3 (-2.45, -0.12), pd = 0.97}} & \textcolor{black}{-3.39 (-8.67, 1.38), pd = 0.89} & \textcolor{black}{-3.26 (-12.57, 4.21), pd = 0.77}\\
\hspace{1em} & Breeding Range Size & \textcolor{black}{0.08 (-0.13, 0.28), pd = 0.75} & \textcolor{black}{0.21 (-0.27, 0.7), pd = 0.77} & \textcolor{black}{0.26 (-0.14, 0.66), pd = 0.87} & \textcolor{black}{0.02 (-0.14, 0.18), pd = 0.58} & \textcolor{black}{0.21 (-0.29, 0.72), pd = 0.77} & \textcolor{black}{0.23 (-0.62, 1.1), pd = 0.69}\\
\hspace{1em} & Phylogenetic Signal λ, Median (90\% Credible Interval) & \textcolor{black}{0.27 (0.15, 0.41)} & \textcolor{black}{0.71 (0.56, 0.85)} & \textcolor{black}{0.6 (0.42, 0.77)} & \textcolor{black}{0.15 (0.07, 0.25)} & \textcolor{black}{0.72 (0.55, 0.86)} & \textcolor{black}{0.85 (0.71, 0.95)}\\
\addlinespace[0.3em]
\multicolumn{1}{l}{\textbf{Breeding Sympatry}}\\
\hspace{1em} & Intercept & \textcolor{black}{-0.9 (-3.45, 1.76), pd = 0.72} & \textcolor{red}{\textbf{-6.89 (-14.7, -0.02), pd = 0.95}} & \textcolor{red}{\textbf{-6.74 (-13.39, -1.09), pd = 0.98}} & \textcolor{red}{\textbf{-1.38 (-3.25, 0.3), pd = 0.91}} & \textcolor{red}{\textbf{-6.34 (-13.61, 0.11), pd = 0.95}} & \textcolor{red}{\textbf{-11.29 (-22.79, -1.24), pd = 0.98}}\\
 & Number of Sympatric Species 
\hspace{1em} (≥ 30\% Breeding Range Overlap) & \textcolor{black}{0.03 (-0.18, 0.24), pd = 0.61} & \textcolor{black}{0.14 (-0.31, 0.56), pd = 0.71} & \textcolor{black}{0.12 (-0.27, 0.49), pd = 0.71} & \textcolor{blue}{\textbf{0.34 (0.17, 0.51), pd = 0.99}} & \textcolor{blue}{\textbf{0.46 (0.01, 0.92), pd = 0.96}} & \textcolor{blue}{\textbf{0.75 (0.03, 1.5), pd = 0.97}}\\
\hspace{1em} & Phylogenetic Signal λ, Median (90\% Credible Interval) & \textcolor{black}{0.26 (0.14, 0.39)} & \textcolor{black}{0.7 (0.54, 0.83)} & \textcolor{black}{0.59 (0.41, 0.77)} & \textcolor{black}{0.13 (0.06, 0.23)} & \textcolor{black}{0.69 (0.52, 0.83)} & \textcolor{black}{0.82 (0.67, 0.94)}\\
\bottomrule
\end{tabular}}
\end{table}
\end{landscape}

\hypertarget{discussion}{%
\section{Discussion}\label{discussion}}

\hypertarget{conclusions}{%
\section{Conclusions}\label{conclusions}}

\hypertarget{acknowledgements}{%
\section{Acknowledgements}\label{acknowledgements}}

\hypertarget{references}{%
\section*{References}\label{references}}
\addcontentsline{toc}{section}{References}

\hypertarget{refs}{}
\begin{CSLReferences}{0}{0}
\leavevmode\hypertarget{ref-andersson2006}{}%
\CSLLeftMargin{1. }
\CSLRightInline{Andersson S, Prager M. 2006 Quantifying {Colors}. In
\emph{Bird coloration, {Volume} 1: {Mechanisms} and {Measurements}} (eds
GE Hill, KJ McGraw), pp. 76--77. {Cambridge, MA}: {Harvard University
Press}. }

\leavevmode\hypertarget{ref-vorobyev1998}{}%
\CSLLeftMargin{2. }
\CSLRightInline{Vorobyev M, Osorio D. 1998 Receptor noise as a
determinant of colour thresholds. \emph{Proceedings. Biological sciences
/ The Royal Society} \textbf{265}, 351--8.
(doi:\href{https://doi.org/10.1098/rspb.1998.0302}{10.1098/rspb.1998.0302})}

\leavevmode\hypertarget{ref-hart2000}{}%
\CSLLeftMargin{3. }
\CSLRightInline{Hart NS, Partridge JC, Cuthill IC, Bennett AT. 2000
Visual pigments, oil droplets, ocular media and cone photoreceptor
distribution in two species of passerine bird: The blue tit ({Parus}
caeruleus {L}.) And the blackbird ({Turdus} merula {L}.). \emph{Journal
of comparative physiology. A, Sensory, neural, and behavioral
physiology} \textbf{186}, 375--387.
(doi:\href{https://doi.org/10.1007/s003590050437}{10.1007/s003590050437})}

\leavevmode\hypertarget{ref-maia2019}{}%
\CSLLeftMargin{4. }
\CSLRightInline{Maia R, Gruson H, Endler JA, White TE. 2019 Pavo 2:
{New} tools for the spectral and spatial analysis of colour in r.
\emph{Methods in Ecology and Evolution} \textbf{10}, 1097--1107.
(doi:\href{https://doi.org/10.1111/2041-210X.13174}{10.1111/2041-210X.13174})}

\leavevmode\hypertarget{ref-rcoreteam2020}{}%
\CSLLeftMargin{5. }
\CSLRightInline{R Core Team. 2020 \emph{R: {A Language} and
{Environment} for {Statistical Computing}}. {Vienna, Austria}: {R
Foundation for Statistical Computing}. }

\leavevmode\hypertarget{ref-endler1993}{}%
\CSLLeftMargin{6. }
\CSLRightInline{Endler JA. 1993 The {Color} of {Light} in {Forests} and
{Its Implications}. \emph{Ecological Monographs} \textbf{63}, 1--27.
(doi:\href{https://doi.org/10.2307/2937121}{10.2307/2937121})}

\leavevmode\hypertarget{ref-kemp2015}{}%
\CSLLeftMargin{7. }
\CSLRightInline{Kemp DJ, Herberstein ME, Fleishman LJ, Endler JA,
Bennett ATD, Dyer AG, Hart NS, Marshall J, Whiting MJ. 2015 An
{Integrative Framework} for the {Appraisal} of {Coloration} in {Nature}.
\emph{The American Naturalist} \textbf{185}, 705--724.
(doi:\href{https://doi.org/10.1086/681021}{10.1086/681021})}

\leavevmode\hypertarget{ref-clement2000}{}%
\CSLLeftMargin{8. }
\CSLRightInline{Clement P, Hathway R. 2000 \emph{Thrushes}. {London}:
{A\&C Black Publishers Ltd}. }

\leavevmode\hypertarget{ref-delhoyo2017}{}%
\CSLLeftMargin{9. }
\CSLRightInline{del Hoyo J, Elliott A, Sargatal J, Christie DA, de Juana
E. 2017 \emph{Handbook of the birds of the world alive}. }

\leavevmode\hypertarget{ref-birdlifeinternationalandhandbookofthebirdsoftheworld2018}{}%
\CSLLeftMargin{10. }
\CSLRightInline{BirdLife International and Handbook of the Birds of the
World. 2018 \emph{Bird species distribution maps of the world. {Version}
2018.1.} }

\leavevmode\hypertarget{ref-vilela2015}{}%
\CSLLeftMargin{11. }
\CSLRightInline{Vilela B, Villalobos F. 2015 {letsR}: A new {R} package
for data handling and analysis in macroecology. \emph{Methods in Ecology
and Evolution} \textbf{6}, 1229--1234.
(doi:\href{https://doi.org/10.1111/2041-210X.12401}{10.1111/2041-210X.12401})}

\leavevmode\hypertarget{ref-cooney2017}{}%
\CSLLeftMargin{12. }
\CSLRightInline{Cooney CR, Tobias JA, Weir JT, Botero CA, Seddon N. 2017
Sexual selection, speciation and constraints on geographical range
overlap in birds. \emph{Ecology Letters} \textbf{20}, 863--871.
(doi:\href{https://doi.org/10.1111/ele.12780}{10.1111/ele.12780})}

\leavevmode\hypertarget{ref-burkner2017}{}%
\CSLLeftMargin{13. }
\CSLRightInline{Bürkner PC. 2017 Brms: {An R} package for {Bayesian}
multilevel models using {Stan}. \emph{Journal of Statistical Software}
\textbf{80}, 1--28.
(doi:\href{https://doi.org/10.18637/jss.v080.i01}{10.18637/jss.v080.i01})}

\leavevmode\hypertarget{ref-nylander2008}{}%
\CSLLeftMargin{14. }
\CSLRightInline{Nylander JAA, Olsson U, Alström P, Sanmartín I. 2008
Accounting for phylogenetic uncertainty in biogeography: {A} bayesian
approach to dispersal-vicariance analysis of the thrushes ({Aves}:
{Turdus}). \emph{Systematic Biology} \textbf{57}, 257--268.
(doi:\href{https://doi.org/10.1080/10635150802044003}{10.1080/10635150802044003})}

\leavevmode\hypertarget{ref-gelman2013}{}%
\CSLLeftMargin{15. }
\CSLRightInline{Gelman A, Carlin JB, Stern HS, Dunson DB, Vehtari A,
Rubin DB. 2013 \emph{Bayesian data analysis, third edition}. Third.
{Boca Raton, FL}: {CRC Press}.
(doi:\href{https://doi.org/10.1201/b16018}{10.1201/b16018})}

\leavevmode\hypertarget{ref-vehtari2017}{}%
\CSLLeftMargin{16. }
\CSLRightInline{Vehtari A, Gelman A, Gabry J. 2017 Practical {Bayesian}
model evaluation using leave-one-out cross-validation and {WAIC}.
\emph{Statistics and Computing} \textbf{27}, 1413--1432.
(doi:\href{https://doi.org/10.1007/s11222-016-9696-4}{10.1007/s11222-016-9696-4})}

\leavevmode\hypertarget{ref-makowski2019}{}%
\CSLLeftMargin{17. }
\CSLRightInline{Makowski D, Ben-Shachar MS, Chen SHA, Lüdecke D. 2019
Indices of {Effect Existence} and {Significance} in the {Bayesian
Framework}. \emph{Frontiers in Psychology} \textbf{10}.
(doi:\href{https://doi.org/10.3389/fpsyg.2019.02767}{10.3389/fpsyg.2019.02767})}

\leavevmode\hypertarget{ref-burnham2002}{}%
\CSLLeftMargin{18. }
\CSLRightInline{Burnham KP, Anderson DR. 2002 \emph{Model selection and
multimodel inference: A practical information-theoretic approach}. 2nd
ed. {New York}: {Springer}. }

\end{CSLReferences}

\end{document}
